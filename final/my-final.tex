%%%%%%%%%%%%%%%%%%%%%%%%%%%%%%%%%%%%%%%%%
% Short Sectioned Assignment
% LaTeX Template
% Version 1.0 (5/5/12)
%
% This template has been downloaded from:
% http://www.LaTeXTemplates.com
%
% Original author:
% Frits Wenneker (http://www.howtotex.com)
%
% License:
% CC BY-NC-SA 3.0 (http://creativecommons.org/licenses/by-nc-sa/3.0/)
%
%%%%%%%%%%%%%%%%%%%%%%%%%%%%%%%%%%%%%%%%%

%----------------------------------------------------------------------------------------
%	PACKAGES AND OTHER DOCUMENT CONFIGURATIONS
%----------------------------------------------------------------------------------------

\documentclass[paper=a4, fontsize=11pt]{scrartcl} % A4 paper and 11pt font size

\usepackage[T1]{fontenc} % Use 8-bit encoding that has 256 glyphs
\usepackage{fourier} % Use the Adobe Utopia font for the document - comment this line to return to the LaTeX default
\usepackage[english]{babel} % English language/hyphenation
\usepackage{amsmath,amsfonts,amsthm} % Math packages

\usepackage{lipsum} % Used for inserting dummy 'Lorem ipsum' text into the template

\usepackage{sectsty} % Allows customizing section commands
\allsectionsfont{\centering \normalfont\scshape} % Make all sections centered, the default font and small caps

\usepackage{fancyhdr} % Custom headers and footers
\pagestyle{fancyplain} % Makes all pages in the document conform to the custom headers and footers
\fancyhead{} % No page header - if you want one, create it in the same way as the footers below
\fancyfoot[L]{} % Empty left footer
\fancyfoot[C]{} % Empty center footer
\fancyfoot[R]{\thepage} % Page numbering for right footer
\renewcommand{\headrulewidth}{0pt} % Remove header underlines
\renewcommand{\footrulewidth}{0pt} % Remove footer underlines
\setlength{\headheight}{13.6pt} % Customize the height of the header

\numberwithin{equation}{section} % Number equations within sections (i.e. 1.1, 1.2, 2.1, 2.2 instead of 1, 2, 3, 4)
\numberwithin{figure}{section} % Number figures within sections (i.e. 1.1, 1.2, 2.1, 2.2 instead of 1, 2, 3, 4)
\numberwithin{table}{section} % Number tables within sections (i.e. 1.1, 1.2, 2.1, 2.2 instead of 1, 2, 3, 4)

\setlength\parindent{0pt} % Removes all indentation from paragraphs - comment this line for an assignment with lots of text

%----------------------------------------------------------------------------------------
%	TITLE SECTION
%----------------------------------------------------------------------------------------

\newcommand{\horrule}[1]{\rule{\linewidth}{#1}} % Create horizontal rule command with 1 argument of height

\title{	
\normalfont \normalsize 
\textsc{University of California San Diego} \\ [25pt] % Your university, school and/or department name(s)
\horrule{0.5pt} \\[0.4cm] % Thin top horizontal rule
\huge CSE 200 Final Exam\\ % The assignment title
\horrule{2pt} \\[0.5cm] % Thick bottom horizontal rule
}

\author{Sai Bi} % Your name

\date{\normalsize\today} % Today's date or a custom date

\begin{document}

\maketitle % Print the title

%----------------------------------------------------------------------------------------
%	PROBLEM 1
%----------------------------------------------------------------------------------------

\section*{Problem 1: Computability of polynomial time}

We prove that PTM is not computable. Otherwise, assume that PTM is computable, then there exists 
a machine $Q(M, x)$ that determines whether $M$ will halt on input $x$. If $M$ halts, $Q$ outputs
$1$, otherwise output $0$.

Now we consider a program $U$ which accepts an input $P$. In the program $U$, it will call $Q(P,
P)$. Following is how $U$ works:
\begin{enumerate}
  \item If $Q(P, P)$ output $1$, $U$ will loop forever.
  \item If $Q(P, P)$ output $0$, $U$ will halt. 
\end{enumerate}

Now we consider the output of $Q(U, U)$. If $Q(U, U)$ output $1$, then $U$ will loop forever, which
contradicts to the definition of $Q(U, U)$ where if $Q(U, U)$ output $1$, it means $U$ will
halt. If $Q(U, U)$ output $0$, then $U$ will halt, which also contradicts to the definition of 
$Q(U, U)$ where if $Q(U, U)$ output $0$, it means $U$ will loop forever.

Therefore we can conclude that PTM is not computable.



\section*{Problem 2: NP-Completeness}
Given graph $G = (V, E)$, we construct a new graph $G^* = (V^*, E^*)$ as following:
\begin{enumerate}
  \item For each node $v_i \in V$, we add a node $v_i^*$ to $G^*$.
  \item For each edge $(v_i, v_j) \in E$, we add a node $v^*_{i, j}$ to $G^*$.
  % \item For each edge $(v_i, v_j) \in E$, we add an edge between $(v_i^*, v_j^*)$ to $G^*$.  
  \item \label{item:1}
    For each node $v^*_{i, j}$ in $G^*$, we add an edge between $v_{i, j}^*$ and $v_i^*$ and
    another edge between $v_{i, j}^*$ and $v_j^*$.
  \item For each pair of nodes like $v_{i, j}^*$ in $G^*$ we add an edge between them. 
\end{enumerate}

With graph $G^*$, we claim that there is an independent set of size $k$ in $G$ if and only if there is a
$3$-separated subset of size $k$ in $G^*$.

\vspace{0.2cm}
\textbf{Claim 1:}
If there is an independent set of size $k$ in $G$, then there is a $3$-separated subset of size $k$ in $G^*$.

\textbf{Proof:}
Let $I = \{v_{a_1}, v_{a_2},\dots, v_{a_k}\}$ be an independent set in $G$. We prove that 
$I^* = \{v_{a_1}^*,  v_{a_2}^*,\dots, v_{a_k}^*\}$ is a $3$-separated subset in $G^*$.

For a pair of node $v^*_{a_i}, v_{a_j}^*$, they are not adjacent, because when constructing $G^*$,
we haven't added any edges between  nodes that are corresponding to original nodes in $G$. Also they
don't share a common neighbour, otherwise according to Step~\ref{item:1}, the common neighbor must
be $V^*_{a_i, a_j}$, which means that there is an edge between $v_{a_i}$ and $v_{a_j}$ in $G$,
which contradicts the fact that $I$ is an independent set. Therefore, each pair of nodes in $I^*$
are not adjacent and don't share a common neighbour, and $I^*$ is a $3$-separated subset of size $k$
in $G^*$.

\vspace{0.2cm}
\textbf{Claim 2:}
If there is a $3$-separated subset of size $k$ in $G^*$, there is an independent set of size $k$ in $G$,   

\textbf{Proof:}
Without loss of generality, we assume that $G^*$ is a connected graph. If it is not, obviously 
$3$-separated subset of each connected component can be combined to form a new $3$-separated subset.
And we only need to consider each connected component separately. 

\vspace{0.2cm}
Let $I^*$ be a $3$-separated subset of size $k$ in $G^*$.  
We first consider the case when all nodes in $I^*$ are in the form of $v^*_{a_i}$. That is,
$I^* = \{v_{a_1}^*,  v_{a_2}^*,\dots, v_{a_k}^*\}$. In this case, we claim that 
$I = \{v_{a_1}, v_{a_2}, \dots, v_{a_k}\}$ is also an independent set of size $k$ in $G$. Otherwise,
assume that $v_{a_i}$ and $v_{a_j}$ are adjacent, it means there is an edge between them. According
to Step~\ref{item:1}, $v*_{a_i}$ and $v^*_{a_j}$  must share a common neighbour $v^*_{a_i, a_j}$,
which contradicts the fact that $v^*_{a_i}$ and $v^*_{a_j}$ are $3$-separated.

\vspace{0.2cm}
Now we consider the case when there is a node $v^*_{a_1, a_2}$ in $I^*$. In this case, we claim that 
$k$ must be $1$. That is, $I^*$ only have a single element $v^*_{a_1, a_2}$. First, it is impossible
to have another $v^*_{a_i, a_j}, i \neq 1 or j \neq 2$, because there is an edge between any two
nodes in the form of $v^*_{a_i, a_j}$. Assume there is another node $v^*_{a_i}$ in $I^*$. Since the
$G^*$ is connected, $v^*_{a_i}$ is connected to some nodes $v^*_{a_i, a_j}$, which must be adjacent
to $v^*_{a_1, a_2}$, which contradicts the fact that $I^*$ is $3$-separated. Therefore, we know that 
$I^*$ has only one element, and we can pick any node in $G$ to form an independent of size $1$.

\vspace{0.2cm}
Since the transform from $G$ to $G^*$ has time complexity $O(|V| + |E|) + O(|E|^2)$,
which is in polynomial time.  
Given Claim 1 and Claim 2, we can show that independent set problem can be reduced to $3$-separated
set problem in polynomial time. 

For any given solution $S$ to $3SS$, first we could check whether 
its size is equal to $k$ in $O(k)$ time. Also to check whether two vertices are $3$-separated, we
could build an adjacent matrix of all vertices in $V$ in $O(|V|^2)$ time, and then for any two
vertex $v_i$ and $v_j$ in $S$, we could check whether they are $3$-separated in $O(|V|)$ time. Hence
in total it takes $O(|V|^3)$ time to check whether $S$ is a valid solution. Therefore, $3SS$ is in
$NP$.

In summary, we have proved that $3SS$ is in NP-Complete. 

\section*{Problem 3: NP-hardness}
We show a reduction to this problem from $3$-SAT. Given a clause in $3$-SAT, we turn it into an
equation as following:
\begin{enumerate}
  \item For a literal $x$, we add a term $(x-1)^2$.
  \item For a literal $\lnot y$, we add a term $y^2$.
\end{enumerate}
Obviously each clause will have $3$ terms. And we multiply them together as a polynomial $P_i$ for
the $i$-th clause.
For example, suppose we have a clause $x \lor \lnot y \lor z$, then the corresponding polynomial
would be $P_i = (x-1)^2 y^2 (z-1)^2$.

After getting the polynomial for each equation, we sum them up to get a single polynomial. Suppose
there are $N$ clauses, and let $P$ be the final polynomial, we have $P = \sum_{i=1}^{N} P_i$. We
claim that the original $3$-SAT problem is satisfiable if and only if there exists a solution to the
function $P = 0$.

\textbf{Claim 1:} If there is a solution to $3$-SAT, there is also a solution to $P = 0$.

\textbf{Proof:}
In the solution to $3$-SAT, if a variable is true, we set the value of corresponding variable in $P$
to $1$ and otherwise we set it to $0$. In this way we get a solution to $P = 0$. The proof is as
following:

For a clause $P_i$ , we know that at least one literal in that clause is true. If
the literal is in the form of $x$, we know that $x$ is true and we have set the value of variable to
$1$ in $P$,  since there is a factor
$(x-1)^2$ in $P_i$, $P_i$ will be zero. If the literal is in the form of $\lnot x$, we have set the
value of variable to $0$ in $P$, since there is a factor $x^2$ in $P_i$, $P_i$ will also be zero. That
is, by assigning values to variables in above way, we will have $P_i = 0$ for all $i$. Therefore we
have got a solution to $P= 0$.

\textbf{Claim 2:} If there is a solution to $P=0$, there is also a solution to $3$-SAT.

\textbf{Proof:}
Since each term $P_i$ is in the form of a product of squares, so $P_i \geq 0$. $P$ is in the form of
a sum of squares, so when $P=0$, $P_i = 0$ for all $i$. If $P_i = 0$, there are two cases:
\begin{enumerate}
  \item there is a factor in the form of $x^2$ equal to zero, in this case, we know that there is
    a literal $\lnot x$ in corresponding clause, we could set value of $x$ to false then the clause
    will be true.
  \item there is a factor in the form of $(x-1)^2$ equal to zero, in this case, we know that there is
    a literal $x$ in corresponding clause, we could set value of $x$ to true then the clause
    will also be true.
\end{enumerate}
For the variables whose value are not set in this process, it means that their values don't affect
the satisfiability of $3$-SAT, they can be set to either true or false.
Therefore in this way we could get a solution to $3$-SAT.

Obviously we know that the above transformation from $3$-SAT to polynomial equations can be done in
polynomial time. Given above claims, we know that the given problem is NP-hard.

This problem might not be in NP because if the input are irrational numbers, then they have infinite
number of digits and we don't have an exact and efficient way to represent it and make a decision whether they
are valid solutions to the problem. Therefore the problem may not be in $NP$.

\section*{Problem 4: Consequences of $P = NP$}
The general idea is that we prove $\epsilon$-median problem is in the class of \textit{bounded-error
probabilistic polynomial time}(BPP) problem.

Since $|y| = n$, so there are $2^n$ different values of $F(x, y)$ on $x$. 
Naively if we want to get the median of $2^n$ elements, we need to check $2^n$ elements one by one.
To avoid this, we uniformly sample a number of elements from $2^n$ elements, and claim that when
this number is large enough, there is a high probability that the median of sampled elements is also
the $\epsilon$-median of original problem.

If we put all $2^n$ values of $F(x, y)$ in an array and sort them in ascending order, we could
divide the array into three parts: the first part $P_1$ is the first $(\frac{1}{2}-\epsilon) 2^n$
elements, the second part $P_2$ is the following $\epsilon 2^n$ elements in the array, and the third
part $P_3$ is the remaining $(\frac{1}{2}-\epsilon) 2^n$ on the right.

Suppose that we are sampling $2k+1$ elements, let $M$ be the median of the sampled elements, there
are three cases here:
\begin{enumerate}
  \item $M \in P_1$. In this case, there must be $k+1$ or more elements sampled from $P_1$. In other
    words, there are no more than $k$ elements sampled from $P_2$ and $P_3$. Let $X$ be the number of
    elements sampled from $P_2$ and $P_3$, we have the following:
    \begin{align}
      \mu[X] & = \frac{1 + \epsilon}{2} (2k+1) \\
      \delta = 1 - \frac{k}{\mu} &= \frac{1 + \epsilon + 2k\epsilon}{(1+\epsilon)(2k+1)} \Rightarrow 
      0 \leq \delta \leq 1\\
      P(X \leq k) &= P(X \leq (1 - \delta) \mu) \leq e^{-\frac{\mu \delta^2}{2}} 
      = e^{-\frac{(u-k)^2}{2u}} (\text{\textit{Chernoff Bounds}}) 
    \end{align}
    That is the probability of $M \in P_1$ is at most $e^{-\frac{(u-k)^2}{2u}}$. 
  \item $M \in P_3$. Similar to the case where $M \in P_1$, we also know that the probability of $M
    \in P_3$ is at most $e^{-\frac{(u-k)^2}{2u}}$. 
  \item $M \in P_2$. We have the following:
    \begin{align}
    P(M \in P_3) &= 1 - P(M \in P_1) - P(M \in P_2) \\
    &\geq 1 - 2 e^{-\frac{(u-k)^2}{2u}}
    \end{align}
\end{enumerate}

To prove that $\epsilon$-median problem is in BPP, we make the probability of median of sampled
elements being also median of median of all elements higher than $\frac{3}{4}$. That is, we need to
guarantee the following:
\begin{align}
   P(M \in P_3) &\geq 1 - 2 e^{-\frac{(u-k)^2}{2u}} \geq \frac{3}{4} \\
   u^2 + k^2 & -2uk -2u\ln 8 \geq 0  
\end{align}
After replace $u$ with $\frac{1 + \epsilon}{2} (2k+1)$, we can see that the coefficient of $k^2$ in 
the equalities is larger than zero, and there must be a solution to the inequality. Also since
$\epsilon \in O(\frac{1}{poly(n)})$, there will exist a $k$ satisfying the inequality with $k \in
O(poly(n))$.

Given above analysis, we have proved that we could uniformly random sample $k \in O(poly(n))$
elements and calculate $F(x, y)$ values in polynomial time, as well as the median of sampled $F$
values in polynomial time. We have proved that the probability of median of sampled elements 
being $\epsilon$-median of the original problem is larger than $\frac{3}{4}$. Therefore, the
original problem is in BPP. If $P = NP$, then we know $P = BPP$. That is, there exist a polynomial
time algorithm to $\epsilon$-median problem.


\end{document}
