\documentclass{article}
\usepackage{amsmath,amsfonts,amsthm,amssymb}
\usepackage{setspace}
\usepackage{fancyhdr}
\usepackage{lastpage}
\usepackage{extramarks}
\usepackage{chngpage}
\usepackage{soul,color}
\usepackage{graphicx,float,wrapfig}
\usepackage{clrscode}
\usepackage{mathrsfs}
\newcommand{\Class}{CSE 200}
% \newcommand{\ClassInstructor}{Russell Impagliazzo}

% Homework Specific Information. Change it to your own
\newcommand{\Title}{Final Exam}
\newcommand{\StudentName}{Sai Bi}
\newcommand{\StudentClass}{}
\newcommand{\StudentNumber}{}

% In case you need to adjust margins:
\topmargin=-0.45in      %
\evensidemargin=0in     %
\oddsidemargin=0in      %
\textwidth=6.5in        %
\textheight=9.0in       %
\headsep=0.25in         %

% Setup the header and footer
\pagestyle{fancy}                                                       %
\lhead{\Title}  %
\rhead{\firstxmark}                                                     %
\lfoot{\lastxmark}                                                      %
\cfoot{}                                                                %
\rfoot{Page\ \thepage\ of\ \protect\pageref{LastPage}}                          %
\renewcommand\headrulewidth{0.4pt}                                      %
\renewcommand\footrulewidth{0.4pt}                                      %

%%%%%%%%%%%%%%%%%%%%%%%%%%%%%%%%%%%%%%%%%%%%%%%%%%%%%%%%%%%%%
% Some tools
\newcommand{\enterProblemHeader}[1]{\nobreak\extramarks{#1}{#1 continued on next page\ldots}\nobreak%
    \nobreak\extramarks{#1 (continued)}{#1 continued on next page\ldots}\nobreak}%
\newcommand{\exitProblemHeader}[1]{\nobreak\extramarks{#1 (continued)}{#1 continued on next page\ldots}\nobreak%
    \nobreak\extramarks{#1}{}\nobreak}%

\providecommand{\myceil}[1]{\left \lceil #1 \right \rceil }
\newcommand{\homeworkProblemName}{}%
\newcounter{homeworkProblemCounter}%
\newenvironment{homeworkProblem}[1][Problem \arabic{homeworkProblemCounter}]%
{\stepcounter{homeworkProblemCounter}%
    \renewcommand{\homeworkProblemName}{#1}%
    \section*{\homeworkProblemName}%
    \enterProblemHeader{\homeworkProblemName}}%
{\exitProblemHeader{\homeworkProblemName}}%

\newcommand{\homeworkSectionName}{}%
\newlength{\homeworkSectionLabelLength}{}%
\newenvironment{homeworkSection}[1]%
{% We put this space here to make sure we're not connected to the above.
    
    \renewcommand{\homeworkSectionName}{#1}%
    \settowidth{\homeworkSectionLabelLength}{\homeworkSectionName}%
    \addtolength{\homeworkSectionLabelLength}{0.25in}%
    \changetext{}{-\homeworkSectionLabelLength}{}{}{}%
    \subsection*{\homeworkSectionName}%
    \enterProblemHeader{\homeworkProblemName\ [\homeworkSectionName]}}%
{\enterProblemHeader{\homeworkProblemName}%
    
    % We put the blank space above in order to make sure this margin
    % change doesn't happen too soon.
    \changetext{}{+\homeworkSectionLabelLength}{}{}{}}%

\newcommand{\Answer}{\ \\\textbf{Answer:} }
\newcommand{\Acknowledgement}[1]{\ \\{\bf Acknowledgement:} #1}
\newcommand{\Complexity}{\vspace{0.3cm} \noindent\textbf{Time Complexity} \\}
\newcommand{\Proof}{\vspace{0.3cm} \noindent\textbf{Proof} \\}
\newcommand{\Algorithm}{\textbf{Algorithm} \\}
\newcommand{\EndProof} { \hfill$\square$ }
\newcommand\equ[1]{\begin{align}\begin{split} #1 \end{split} \end{align}}
%%%%%%%%%%%%%%%%%%%%%%%%%%%%%%%%%%%%%%%%%%%%%%%%%%%%%%%%%%%%%
\setlength\parindent{0pt}
\setlength{\parskip}{0.2cm}%% 
\usepackage{minted}
%%%%%%%%%%%%%%%%%%%%%%%%%%%%%%%%%%%%%%%%%%%%%%%%%%%%%%%%%%%%%
% Make title
\title{\textmd{\bf \Class: \Title}}
\date{}
\author{\textbf{\StudentName}}
%%%%%%%%%%%%%%%%%%%%%%%%%%%%%%%%%%%%%%%%%%%%%%%%%%%%%%%%%%%%%

\begin{document}
\maketitle \thispagestyle{empty}
\section*{Problem 2: NP-Completeness}
First we prove that $3SS \in NP$. For any given solution $S$ to $3SS$, first we could check whether 
its size is equal to $k$ in $O(k)$ time. Also to check whether two vertices are $3$-separated, we
could build an adjacent matrix of all vertices in $V$ in $O(|V|^2)$ time, and then for any two
vertex $v_i$ and $v_j$ in $S$, we could check whether they are $3$-separated in $O(|V|)$ time. Hence
in total it takes $O(|V|^3)$ time to check whether $S$ is a valid solution. Therefore, $3SS$ is in
$NP$.    

Following we give a reduction from \textit{Independent Set} problem.

Given a graph $G = (V, E)$, we reconstruct a graph $G' = (V', E')$ as following:
\begin{enumerate}
  \item For each edge $e_k = (v_i, v_j) \in E$, we add nodes $v_{i, e_k}, v_{j, e_i}$ to $G'$, and add an edge
    between them.
  \item If a node $v_i \in V$ is not involved in any edges, we add node $v_i'$ to $G'$.   
  \item For each pair of nodes in $v_{i, e_{a_1}}, \dots, v_{i, e_{a_k}}$, where $a_1, \dots, a_k$
    are indices of edges,  we add an edge bewteen them.
\end{enumerate}

\textbf{Claim 1:} If there is an independent set of size $k$ in $G$, then there also exists a
$3$-separated set of size $k$ in $G'$.

\textbf{Proof:}
Suppose that we have an independent set $A = \{v_1, v_2, \dots, v_k \}$ in $G$, we can construct a
$3$-separated set $A'$ of $G'$ as following:
\begin{enumerate}
  \item If $v_i \in A$ is not involved in any edges in $G$, we know that $v_i'$ is in $V'$, we add $v_i$ to $V'$.
  \item For any other $v_i \in A$, suppose $v_i$ is an endpoint of $e_k$, we add vertex $v_{i, e_k}$
    to $A'$. Note that for each $v_i$, we only add a single vertex to $A'$.
\end{enumerate}

We prove that $A'$ is a $3$-separated subset. Obviously $A'$ has a size of $k$. For $v_i' \in A'$,
we know that $v_i'$ is not adjacent to any other vertices in $A'$, so $v_i'$ is $3$-separated from
any other vertices in $A'$.

For $v_{i, e_p} \in A'$, and $v_{j, e_q} \in A', i \neq j$, they are also $3$-separated. 
Assume that they are not $3$-separated,  there are two cases here:
\begin{enumerate}
  \item They are adjacent to each other. Since $i \neq j$, from the way $G'$ is constructed, we know that $v_i$ and
    $v_j$ must be adjacent to each other, which contradicts the fact that $A$ is an independent set.
  \item They have a common neighbour. There are also two cases here. First, assume that they have a common
    neighbour $v_{i, e_o}, o \neq p$. Since $v_{i, e_o}$ is adjacent to $v_{j, e_q}$, similar to
    above we know that $v_i$ and $v_j$ are adjacent in $G$, which forms a contradiction. The same
    holds when the common neighbour is $v_{j, e_o}$.

    For second case, assume that they have a common neighbour $v_{m, e_o}, m\neq i,m \neq j $, 
    obviously we have $e_p = e_o, e_q = e_o$, that is, $e_p = e_q$. From the way $G'$ is
    constructed, we know that $v_i$ and $v_j$ is adjacent, thus forming a contradiction.     
\end{enumerate}
In summary, we know that $A'$ is a $3$-separated subset of size $k$.

\textbf{Claim 2:} If there is a $3$-separated set of size $k$ in $G'$, there is also an independent
of size $k$ in $G$.

\textbf{Proof:} Suppose that $A'$ is a $3$-separated set in $G'$, we can get an independent $A$ as
following:
\begin{enumerate}
  \item If $v_i'\in A'$, that is $v_i'$ is not connected by any edges, we add $v_i$ to $A$.
  \item If $v_{i, e_p}\in A'$, we add  $v_i$ to $A$.
\end{enumerate}
We prove as following: first, if $v_i' \in A'$, then $v_i$ is not adjacent to any other vertices in
$V$, therefore we can safely add $v_i$ to $A$ without violating requirements of independent set.

For any $v_{i, e_p} \in A', v_{j, e_q} \in A'$, if $i = j$, obviously they are adjacent to each
other, so they cannot be in $A'$ at the same time. If $i \neq j$, we must have $v_i$ and $v_j$ are
not adjacent. Otherwise suppose they are adjacent, since all $v_{i, e_{a_i}}$ are connected to each
other, and all $v_{j, e_{a_i}}$ are connected to each other, and there is an edge between some
$v_{i, e_{a_s}}$ and some $v_{j, e_{a_t}}$, so $v_{i, e_p}$ and $v_{j, e_q}$ must either be adjacent
or 



\end{document}





