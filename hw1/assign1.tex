\documentclass[a4paper,11pt]{article}

\usepackage[T1]{fontenc}
\usepackage[utf8]{inputenc}
\usepackage{graphicx}
\usepackage{xcolor}
 \usepackage{tgtermes}

 \usepackage[
 pdftitle={Math Assignment},
 pdfauthor={Joe Doe, Some University},
 colorlinks=true,linkcolor=blue,urlcolor=blue,citecolor=blue,bookmarks=true,
 bookmarksopenlevel=2]{hyperref}
\usepackage{amsmath,amssymb,amsthm,textcomp}
\usepackage{enumerate}
\usepackage{multicol}
\usepackage{tikz}

\usepackage{geometry}
\geometry{total={210mm,297mm},
left=25mm,right=25mm,%
bindingoffset=0mm, top=20mm,bottom=20mm}


\linespread{1.3}

\newcommand{\linia}{\rule{\linewidth}{0.5pt}}

\providecommand{\myceil}[1]{\left \lceil #1 \right \rceil }
% custom theorems if needed
\newtheoremstyle{mytheor}
    {1ex}{1ex}{\normalfont}{0pt}{\scshape}{.}{1ex}
    {{\thmname{#1 }}{\thmnumber{#2}}{\thmnote{ (#3)}}}

\theoremstyle{mytheor}
\newtheorem{defi}{Definition}
\usepackage[ruled, vlined, linesnumbered,lined,boxed,commentsnumbered]{algorithm2e}
\usepackage[parfill]{parskip}
% my own titles
\makeatletter
% \renewcommand{\maketitle}{
% \begin{center}
% \vspace{2ex}
% {\huge \textsc{\@title}}
% \vspace{1ex}
% \\
% \linia\\
% \@author \hfill \@date
% \vspace{4ex}
% \end{center}
% }
% \makeatother
%%%
\setlength\parindent{0pt}
% custom footers and headers
\usepackage{fancyhdr,lastpage}
% \pagestyle{fancy}
% \lhead{}
% \chead{}
% \rhead{}
% \lfoot{}
% \cfoot{}
% \rfoot{Page \thepage\ /\ \pageref*{LastPage}}
% \renewcommand{\headrulewidth}{0pt}
% \renewcommand{\footrulewidth}{0pt}
%

%%%----------%%%----------%%%----------%%%----------%%%

\begin{document}

\title{CSE 200: Complexity}

\author{Sai Bi}

\date{\today}

\maketitle

\section*{Problem 1}
Given $f(n+1) \in O(f(n))$, then there exists a positive number $M$ and a positive integer $n_0$ satisfying
\begin{equation}
  f(n+1) \leq M|f(n)|, \text{for all } n \geq n_0
\end{equation}
Hence for all positive integers $n \geq n_0$, we have
\begin{align}
  \begin{split}
  f(n) & \leq M^{n - n_0} f(n_0)\\
       & = 2^{(n-n_0)\log_2{M}} f(n_0) \\
       & = M' |2^{cn}|, \text{where } c = \log_2{M}, M' = f(n_0) M^{-n_0}
  \end{split}
\end{align}
Therefore, we know that $f(n) \in O(2^{cn})$

The inverse is not ture. An counter-example is $f(n) = 2^{2^{\myceil{\log n}}}$. Let $n$ be a power of $2$, then
we have $f(n) = 2^{2^{\log n}} = 2^n, f(n+1) = 2^{2^{\myceil{\log (n+1)}}} = 2^{2^{\log n + 1}} = 2^{2n}$. In this case,
we have $f(n+1) = 2^n f(n)$. Then for any $M > 0$ and positive integer $n_0$, let $n$ be a power of $2$ and $n > \log M, n > n_0$,
we will have $f(n+1) = 2^n f(n) > M f(n)$. That is, $f(n+1) \notin O(f(n))$.

\end{document}
